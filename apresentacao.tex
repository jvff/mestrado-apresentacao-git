\documentclass{beamer}

\usepackage[utf8]{inputenc}
\usepackage[brazil]{babel}

\mode<presentiation>{\usetheme{boxes}}

\setbeamertemplate{footline}[frame number]

\newenvironment{slide}{\begin{frame}{\insertsection}}{\end{frame}}

\title{Introdução ao Git}
\author{Janito Vaqueiro Ferreira Filho}
\institute{Universidade Estadual de Campinas}

\begin{document}

\begin{frame}
    \titlepage
\end{frame}

\begin{frame}
    \frametitle{Agenda}
    \tableofcontents
\end{frame}

\section{Introdução}
\begin{slide}
    \begin{itemize}
        \item Todo projeto contém artefatos
        \pause
        \item A complexidade de organizar esses artefatos depende de:
        \pause
            \begin{itemize}
                \item Número de artefatos
                \pause
                \item Sujeitabilidade a mudanças
                \pause
                \item Número de pessoas envolvidas
            \end{itemize}
    \end{itemize}
\end{slide}

\section{Motivação}
\begin{slide}
    A organização incorreta pode resultar em diversos problemas
    \begin{itemize}
        \pause
        \item Perda de artefatos
        \pause
        \item Perda da rastreabilidade
        \pause
        \item Perda da reprodutibilidade
        \pause
        \item Perda de coerência no uso dos artefatos entre os envolvidos
    \end{itemize}
\end{slide}

\end{document}
