\documentclass{beamer}

\usepackage[utf8]{inputenc}
\usepackage[brazil]{babel}

\mode<presentiation>{\usetheme{boxes}}

\setbeamertemplate{footline}[frame number]

\newenvironment{slide}{\begin{frame}{\insertsection}}{\end{frame}}

\title{Introdução ao Git}
\author{Janito Vaqueiro Ferreira Filho}
\institute{Universidade Estadual de Campinas}

\begin{document}

\begin{frame}
    \titlepage
\end{frame}

\begin{frame}
    \frametitle{Agenda}
    \tableofcontents
\end{frame}

\section{Introdução}
\begin{slide}
    \begin{itemize}
        \item Todo projeto contém artefatos
        \pause
        \item A complexidade de organizar esses artefatos depende de:
        \pause
            \begin{itemize}
                \item Número de artefatos
                \pause
                \item Sujeitabilidade a mudanças
                \pause
                \item Número de pessoas envolvidas
            \end{itemize}
    \end{itemize}
\end{slide}

\section{Motivação}
\begin{slide}
    A organização incorreta pode resultar em diversos problemas
    \begin{itemize}
        \pause
        \item Perda de artefatos
        \pause
        \item Perda da rastreabilidade
        \pause
        \item Perda da reprodutibilidade
        \pause
        \item Perda de coerência no uso dos artefatos entre os envolvidos
    \end{itemize}
\end{slide}

\section{Solução}
\begin{slide}
    Todas as soluções estão relacionadas a dois pontos chaves
    \begin{itemize}
        \pause
        \item Aumento da documentação
        \begin{itemize}
            \pause
            \item Rotular os artefatos
            \pause
            \item Descrever os artefatos
            \pause
            \item Registrar a evolução dos artefatos
        \end{itemize}
        \pause
        \item Aumento da redundância
        \begin{itemize}
            \pause
            \item Manter cópias dos artefatos
            \pause
            \item Manter cópias do estado dos artefatos em instantes de tempo
                \pause (versões dos artefatos)
            \pause
            \item Manter cópias da documentação e dos registros
        \end{itemize}
    \end{itemize}
\end{slide}

\section{Sistema de Controle de Versionamento}
\begin{slide}
    \begin{itemize}
        \item Ferramenta para auxiliar na organização de artefatos digitais
            \pause (arquivos)
        \pause
        \item Foco em controle de versões dos artefatos
        \pause
        \item Foco em facilitar o uso por diversas pessoas
        \pause
        \item \textbf{Repositórios} são conjuntos de artefatos e registros de um
            mesmo projeto organizados de forma unificada
    \end{itemize}
\end{slide}

\end{document}
