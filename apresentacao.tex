\documentclass{beamer}

\usepackage[utf8]{inputenc}
\usepackage[brazil]{babel}
\usepackage{tikz}

\usetikzlibrary{calc}
\usetikzlibrary{fit}
\usetikzlibrary{intersections}
\usetikzlibrary{positioning}
\usetikzlibrary{shapes}

\tikzset{commit/.style={draw,circle,minimum size=7mm}}

\mode<presentiation>{\usetheme{boxes}}

\setbeamertemplate{footline}[frame number]

\newenvironment{slide}{\begin{frame}{\insertsection}}{\end{frame}}
\newenvironment{ctikz}%
    {\begin{center}\begin{tikzpicture}}%
    {\end{tikzpicture}\end{center}}

\title{Introdução ao Git}
\author{Janito Vaqueiro Ferreira Filho}
\institute{Universidade Estadual de Campinas}

\begin{document}

\begin{frame}
    \titlepage
\end{frame}

\begin{frame}
    \frametitle{Agenda}
    \tableofcontents
\end{frame}

\section{Introdução}
\begin{slide}
    \begin{itemize}
        \item Todo projeto contém artefatos
        \pause
        \item A complexidade de organizar esses artefatos depende de:
        \pause
            \begin{itemize}
                \item Número de artefatos
                \pause
                \item Frequência de mudanças
                \pause
                \item Número de pessoas envolvidas
            \end{itemize}
    \end{itemize}
\end{slide}

\section{Motivação}
\begin{slide}
    A organização incorreta pode resultar em diversos problemas
    \begin{itemize}
        \pause
        \item Perda de artefatos
        \pause
        \item Perda da rastreabilidade
        \pause
        \item Perda da reprodutibilidade
        \pause
        \item Perda de coerência no uso dos artefatos entre os envolvidos
    \end{itemize}
\end{slide}

\section{Solução}
\begin{slide}
    Todas as soluções estão relacionadas a dois pontos chaves
    \begin{itemize}
        \pause
        \item Aumento da documentação
        \begin{itemize}
            \pause
            \item Rotular os artefatos
            \pause
            \item Descrever os artefatos
            \pause
            \item Registrar a evolução dos artefatos
        \end{itemize}
        \pause
        \item Aumento da redundância
        \begin{itemize}
            \pause
            \item Manter cópias dos artefatos
            \pause
            \item Manter cópias dos estados dos artefatos em instantes de tempo
                \pause (versões dos artefatos)
            \pause
            \item Manter cópias da documentação e dos registros
        \end{itemize}
    \end{itemize}
\end{slide}

\section{Sistema de Controle de Versionamento}
\begin{slide}
    \begin{itemize}
        \item Ferramenta para auxiliar na organização de artefatos digitais
            \pause (arquivos)
        \pause
        \item Foco em controle de versões dos artefatos
        \pause
        \item Foco em facilitar o uso por diversas pessoas
        \pause
        \item \textbf{Repositórios} são conjuntos de artefatos e registros de um
            mesmo projeto organizados de forma unificada
        \pause
        \item Existem diversos sistemas
        \begin{itemize}
            \pause
            \item CVS
            \pause
            \item Subversion (SVN)
            \pause
            \item Mercurial (Hg)
            \pause
            \item Git
        \end{itemize}
    \end{itemize}
\end{slide}

\section{Git}
\begin{slide}
    \begin{itemize}
        \item Feito pelo o Linus Torvalds para organizar o repositório do Linux
        \pause
        \item Descentralizado
        \begin{itemize}
            \pause
            \item Cada pessoa (ou computador) contém uma cópia completa do
                repositório
            \pause
            \item Não requer conexão à internet
        \end{itemize}
        \pause
        \item Versionamento do conjunto de artefatos como um todo
        \begin{itemize}
            \pause
            \item Não há necessidade de controlar o versionamento de cada
                arquivo de forma individual
        \end{itemize}
        \pause
        \item Projetado para melhor funcionamento com arquivos textuais
        \begin{itemize}
            \pause
            \item Algumas funcionalidades são perdidas se arquivos binários
                forem utilizados \pause (.exe, .pdf, .zip, ...)
        \end{itemize}
    \end{itemize}
\end{slide}

\section{Instalação}
\begin{slide}
    \begin{itemize}
        \item Linux
        \begin{itemize}
            \pause
            \item Ubuntu e Debian: \texttt{sudo apt-get install git}
            \pause
            \item Fedora e CentOS: \texttt{sudo yum install git}
            \pause
            \item Outros (e UNIX): \texttt{https://git-scm.com/download/linux}
        \end{itemize}
        \pause
        \item Windows
        \begin{itemize}
            \pause
            \item Existem diversos programas
            \pause
            \item Sugestão: Git for Windows Portable
            \begin{itemize}
                \pause
                \item \texttt{https://git-scm.com/download/win}
                \pause
                \item Mais próximo do ambiente Linux
            \end{itemize}
        \end{itemize}
        \pause
        \item Após a instalação, é preciso se identificar
        \begin{itemize}
            \pause
            \item \texttt{git config --global user.name "Nome Completo"}
            \item \texttt{git config --global user.email "email@local.com"}
        \end{itemize}
    \end{itemize}
\end{slide}

\section{Conceitos Básicos}
\begin{slide}
    \begin{itemize}
        \item O Git controla um diretório contendo o repositório
        \begin{itemize}
            \pause
            \item O comando \texttt{git init} cria um repositório em uma pasta
        \end{itemize}
        \pause
        \item Dentro da pasta, ele cria uma pasta oculta \emph{.git}
        \begin{itemize}
            \pause
            \item Cópias dos artefatos, dos registros e arquivos internos
        \end{itemize}
        \pause
        \item O Git irá controlar os arquivos da pasta que são artefatos
        \begin{itemize}
            \pause
            \item O arquivo \emph{.gitignore} pode ser usado para definir quais
                arquivos não são artefatos
            \begin{itemize}
                \pause
                \item Arquivos temporários
                \pause
                \item Arquivos gerados a partir dos artefatos
                \pause
                \item Arquivos locais (específicos ao computador atual)
                \pause
                \item Arquivos de segurança (com senhas, por exemplo)
            \end{itemize}
        \end{itemize}
    \end{itemize}
\end{slide}

\section{Controlando o Diretório}
\begin{slide}
    \begin{itemize}
        \item Como o Git controla o diretório do repositório, é preciso ter
            alguns cuidados
        \begin{itemize}
            \pause
            \item Requisitar ao Git para voltar para uma versão anterior
                sobrescreve os artefatos
            \pause
            \item A ideia é restaurar a pasta para o estado anterior exatamente
                como ela estava
            \pause
            \item Se as mudanças não forem versionadas antes de realizar este
                tipo de operação, alterações podem ser perdidas
            \pause
            \item Se em uma versão anterior um dado artefato não existia, o
                arquivo será apagado quando se voltar àquela versão
        \end{itemize}
    \end{itemize}
\end{slide}

\section{Índice}
\begin{slide}
    \begin{itemize}
        \item Um local temporário de cópias de artefatos
        \pause
        \item Os arquivos no índice serão usados para formar uma nova versão
        \pause
        \item Comandos importantes
        \begin{itemize}
            \pause
            \item \texttt{git status}
            \pause
            \item \texttt{git add "arquivo"}
            \pause
            \item \texttt{git reset -- "arquivo"}
            \pause
            \item \texttt{git rm "arquivo"}
            \pause
            \item \texttt{git diff --cached}
        \end{itemize}
    \end{itemize}
\end{slide}

\section{Commit}
\begin{slide}
    \begin{itemize}
        \item Representa uma versão do repositório
        \pause
        \item Um \emph{commit} é nomeado pelo seu \emph{hash} SHA-1
        \pause
        \item Referencia uma cópia do estado de todos os artefatos em um
            determinando instante de tempo
        \pause
        \item Informações que compõem um \emph{commit}
        \begin{itemize}
            \pause
            \item Cópia dos artefatos \pause (estado)
            \pause
            \item Título e descrição \pause (porquê)
            \pause
            \item Autor \pause (quem)
            \pause
            \item Data \pause (quando)
            \pause
            \item \emph{Commit} pai \pause (rastreabilidade)
        \end{itemize}
        \pause
        \item Comandos importantes
        \begin{itemize}
            \pause
            \item \texttt{git commit}
            \pause
            \item \texttt{git show}
            \pause
            \item \texttt{git checkout}
        \end{itemize}
    \end{itemize}
\end{slide}

\section{Ramificação}
\begin{slide}
    \begin{itemize}
        \item Na teoria, uma ramificação (\emph{branch}) é uma sequência de
            \emph{commits}
    \end{itemize}
    \begin{ctikz}
        \node[commit,color=blue] (a1) {};
        \node[commit,color=blue] (a2) [right=of a1] {};
        \node[commit,color=blue] (a3) [right=of a2] {};
        \node[commit,color=blue] (a4) [right=of a3] {};
        \node[commit,color=red] (b1) [below right=of a1] {};
        \node[commit,color=red] (b2) [right=of b1] {};
        \node[commit,color=red] (b3) [right=of b2] {};
        \node[commit,color=red] (b4) [right=of b3] {};

        \draw [thick,blue,<-] (a1) -- (a2);
        \draw [thick,blue,<-] (a2) -- (a3);
        \draw [thick,blue,<-] (a3) -- (a4);
        \draw [thick,red,<-] (a1) -- (b1);
        \draw [thick,red,<-] (b1) -- (b2);
        \draw [thick,red,<-] (b2) -- (b3);
        \draw [thick,red,<-] (b3) -- (b4);
    \end{ctikz}
\end{slide}

\begin{slide}
    \begin{itemize}
        \item Na prática é um "ponteiro ambulante"
        \pause
        \begin{ctikz}
            \node[commit] (a1) {};
            \node[commit] (a2) [right=of a1] {};
            \node[commit] (a3) [right=of a2] {};
            \node[commit] (a4) [right=of a3] {};
            \node[commit] (b1) [below right=of a1] {};
            \node[commit] (b2) [right=of b1] {};
            \node[commit] (b3) [right=of b2] {};
            \node[commit] (b4) [right=of b3] {};

            \draw [thick,<-] (a1) -- (a2);
            \draw [thick,<-] (a2) -- (a3);
            \draw [thick,<-] (a3) -- (a4);
            \draw [thick,<-] (a1) -- (b1);
            \draw [thick,<-] (b1) -- (b2);
            \draw [thick,<-] (b2) -- (b3);
            \draw [thick,<-] (b3) -- (b4);

            \draw [thick,dashed,blue,<-] (a4) -- ++(0.7,0.7)
                node [right] {ramificação1};
            \draw [thick,dashed,red,<-] (b4) -- ++(0.7,0.7)
                node [right] {ramificação2};
        \end{ctikz}
        \pause
        \item A cada novo \emph{commit} acrescentado à ramificação, o
            ponteiro é atualizado automaticamente
        \pause
        \begin{ctikz}
            \node[commit,gray] (a1) {};
            \node[commit,gray] (a2) [right=of a1] {};
            \node[commit,gray] (a3) [right=of a2] {};
            \node[commit,gray] (a4) [right=of a3] {};
            \node[commit] (a5) [right=of a4] {};
            \node[commit,gray] (b1) [below right=of a1] {};
            \node[commit,gray] (b2) [right=of b1] {};
            \node[commit,gray] (b3) [right=of b2] {};
            \node[commit,gray] (b4) [right=of b3] {};

            \draw [thick,gray,<-] (a1) -- (a2);
            \draw [thick,gray,<-] (a2) -- (a3);
            \draw [thick,gray,<-] (a3) -- (a4);
            \draw [thick,<-] (a4) -- (a5);
            \draw [thick,gray,<-] (a1) -- (b1);
            \draw [thick,gray,<-] (b1) -- (b2);
            \draw [thick,gray,<-] (b2) -- (b3);
            \draw [thick,gray,<-] (b3) -- (b4);

            \draw [thick,dashed,blue,<-] (a5) -- ++(0.7,0.7)
                node [right] {ramificação1};
            \draw [thick,dashed,red,<-] (b4) -- ++(0.7,0.7)
                node [right] {ramificação2};
        \end{ctikz}
    \end{itemize}
\end{slide}

\begin{slide}
    \begin{itemize}
        \item Comandos (iniciais) importantes
        \begin{itemize}
            \pause
            \item \texttt{git branch "nova\_ramificacao"}
            \pause
            \item \texttt{git checkout "ramificacao"}
            \pause
            \item \texttt{git checkout -b "outra\_nova\_ramificacao"}
        \end{itemize}
        \pause
        \item Todo repositório começa com uma ramificação inicial chamada
            \textbf{master}
    \end{itemize}
\end{slide}

\section{Junção de Ramificações}
\begin{slide}
    \begin{itemize}
        \item Uma utilidade de usar ramificações é poder juntá-las
            posteriormente
    \end{itemize}
    \begin{ctikz}
        \node[commit,color=blue] (a1) {};
        \node[commit,color=blue] (a2) [right=of a1] {};
        \node[commit,color=blue] (a3) [right=of a2] {};
        \node[commit,color=red] (b1) [below right=of a1] {};
        \node[commit,color=red] (b2) [right=of b1] {};
        \node[commit,color=red] (b3) [right=of b2] {};

        \node (j) [above right=0.5 of b3] {?};

        \draw [thick,blue,<-] (a1) -- (a2);
        \draw [thick,blue,<-] (a2) -- (a3);
        \draw [thick,red,<-] (a1) -- (b1);
        \draw [thick,red,<-] (b1) -- (b2);
        \draw [thick,red,<-] (b2) -- (b3);

        \draw [dotted,->] (a3) -- (j);
        \draw [dotted,->] (b3) -- (j);
    \end{ctikz}
\end{slide}

\begin{slide}
    \begin{itemize}
        \item A maneira mais fácil de fazer isso é usando o comando
        \begin{itemize}
            \pause
            \item \texttt{git merge ramificacao2}
        \end{itemize}
        \pause
        \begin{ctikz}
            \node[commit,color=gray] (a1) {};
            \node[commit,color=gray] (a2) [right=of a1] {};
            \node[commit,color=gray] (a3) [right=of a2] {};
            \node[commit] (aj) [right=of a3] {};
            \node[commit,color=gray] (b1) [below right=of a1] {};
            \node[commit,color=gray] (b2) [right=of b1] {};
            \node[commit,color=gray] (b3) [right=of b2] {};

            \draw [thick,gray,<-] (a1) -- (a2);
            \draw [thick,gray,<-] (a2) -- (a3);
            \draw [thick,gray,<-] (a1) -- (b1);
            \draw [thick,gray,<-] (b1) -- (b2);
            \draw [thick,gray,<-] (b2) -- (b3);

            \draw [thick,<-] (a3) -- (aj);
            \draw [thick,<-] (b3) -- (aj);

            \draw [thick,dashed,blue,<-] (aj) -- ++(0.7,0.7)
                node [right] {ramificação1};
            \draw [thick,dashed,red,<-] (b3) -- ++(0.7,0.7)
                node [right] {ramificação2};
        \end{ctikz}
        \pause
        \item O comando cria um \emph{commit} especial
        \pause
        \item A diferença é que este \emph{commit} tem dois pais
        \begin{itemize}
            \pause
            \item O primeiro é o \emph{commit} atual
            \pause
            \item O segundo é o \emph{commit} juntado
        \end{itemize}
        \pause
        \item Observações
        \begin{itemize}
            \pause
            \item Junta \emph{commits} ao invés de ramificações
            \pause
            \item A ramificação juntada não é apagada
            \pause
            \item A ramificação juntada pode continuar a evoluir posteriormente,
                possivelmente necessitando que outra junção futura
        \end{itemize}
    \end{itemize}
\end{slide}

\section{Junção Sem Commit Especial}
\begin{slide}
    \begin{itemize}
        \item Uma forma de juntar sem o commit especial é refazer todos os
            \emph{commits} de uma ramificação em outra
    \end{itemize}
    \begin{ctikz}
        \node[commit,color=blue] (a1) {};
        \node[commit,color=blue] (a2) [right=of a1] {};
        \node[commit,color=blue] (a3) [right=of a2] {};
        \node[commit,color=gray] (b1) [below right=of a1] {};
        \node[commit,color=gray] (b2) [right=of b1] {};
        \node[commit,color=gray] (b3) [right=of b2] {};
        \node[commit,color=red] (nb1) [right=of a3] {};
        \node[commit,color=red] (nb2) [right=of nb1] {};
        \node[commit,color=red] (nb3) [right=of nb2] {};

        \draw [thick,blue,<-] (a1) -- (a2);
        \draw [thick,blue,<-] (a2) -- (a3);
        \draw [thick,gray,<-] (a1) -- (b1);
        \draw [thick,gray,<-] (b1) -- (b2);
        \draw [thick,gray,<-] (b2) -- (b3);
        \draw [thick,red,<-] (a3) -- (nb1);
        \draw [thick,red,<-] (nb1) -- (nb2);
        \draw [thick,red,<-] (nb2) -- (nb3);
    \end{ctikz}
\end{slide}

\begin{slide}
    \begin{itemize}
        \item Existe um comando que auxilia neste processo
        \begin{itemize}
            \pause
            \item \texttt{git rebase "outra\_ramificacao"}
        \end{itemize}
        \pause
        \item Refaz todos os \emph{commits} da ramificação atual como se
            tivessem começado após o \emph{commit} especificado
        \pause
        \item A junção é finalizada de forma semelhante à junção normal, porém o
            Git detecta que não há necessidade de criar o \emph{commit} especial
        \pause
        \item Exemplo
        \begin{itemize}
            \pause
            \item \texttt{git checkout ramificacao}
            \pause
            \item \texttt{git rebase master}
            \pause
            \item \texttt{git checkout master}
            \pause
            \item \texttt{git merge ramificacao}
        \end{itemize}
    \end{itemize}
\end{slide}

\section{Junção Avançada}
\begin{slide}
    \begin{itemize}
        \item Existe uma versão interativa do comando \emph{rebase}\pause:
        \begin{itemize}
            \item \texttt{git rebase -i "ramificacao"}
        \end{itemize}
        \pause
        \item Ele permite alterar as operações que serão realizadas
    \end{itemize}
    \begin{ctikz}
        \node[commit,color=blue] (a1) {};
        \node[commit,color=blue] (a2) [right=of a1] {};
        \node[commit,color=blue] (a3) [right=of a2] {};
        \node[commit,color=gray] (b1) [below right=of a1] {1};
        \node[commit,color=gray] (b2) [right=of b1] {2};
        \node[commit,color=gray] (b3) [right=of b2] {3};
        \node[commit,color=gray] (b4) [right=of b3] {4};
        \node[commit,color=gray] (b5) [right=of b4] {5};
        \node[commit,color=red] (nb1) [right=of a3] {4};
        \node[commit,color=red] (nb2) [right=of nb1] {1\&2};
        \node[commit,color=red] (nb3) [right=of nb2] {5*};

        \draw [thick,blue,<-] (a1) -- (a2);
        \draw [thick,blue,<-] (a2) -- (a3);
        \draw [thick,gray,<-] (a1) -- (b1);
        \draw [thick,gray,<-] (b1) -- (b2);
        \draw [thick,gray,<-] (b2) -- (b3);
        \draw [thick,gray,<-] (b3) -- (b4);
        \draw [thick,gray,<-] (b4) -- (b5);
        \draw [thick,red,<-] (a3) -- (nb1);
        \draw [thick,red,<-] (nb1) -- (nb2);
        \draw [thick,red,<-] (nb2) -- (nb3);
    \end{ctikz}
    \begin{itemize}
        \pause
        \item Alterar a ordem dos \emph{commits}
        \pause
        \item Pular \emph{commits}
        \pause
        \item Re-escrever a descrição de \emph{commits}
        \pause
        \item Juntar dois ou mais \emph{commits} em um só
        \pause
        \item Um dos comandos mais poderosos
    \end{itemize}
\end{slide}

\section{Resolução de Conflitos}
\begin{slide}
    \begin{itemize}
        \item Pode ocorrer de um mesmo arquivo ser alterado por dois
            \emph{commits} diferentes que serão juntados
        \pause
        \item O Git tentará resolver os conflitos automaticamente
        \pause
        \item Se não conseguir, indicará no arquivo os problemas
        \begin{itemize}
            \pause
            \item \texttt{<<<<<<<}
            \item \texttt{Lorem ipsum dolor sit amet, consectetur adipiscing
                elit.}
            \item \texttt{=======}
            \item \texttt{Nam fermentum tristique massa sit amet dictum.}
            \item \texttt{>>>>>>>}
        \end{itemize}
        \pause
        \item Basta resolver os problemas e adicionar ao índice
        \begin{itemize}
            \pause
            \item Possivelmente rodar \texttt{git rebase --continue}
        \end{itemize}
    \end{itemize}
\end{slide}

\section{Visualizando o Histórico}
\begin{slide}
    \begin{itemize}
        \item O comando \texttt{git log} percorre todos os \emph{commits} a
            partir do \emph{commit} atual
        \pause
        \item Algumas opções podem ser úteis
        \begin{itemize}
            \pause
            \item \texttt{--oneline}
            \begin{itemize}
                \item Mostra somente o título
            \end{itemize}
            \pause
            \item \texttt{--decorate}
            \begin{itemize}
                \item Mostra os "ponteiros" encontrados no caminho
            \end{itemize}
            \pause
            \item \texttt{--all}
            \begin{itemize}
                \item Começa a percorrer a partir de todos os ponteiros, na
                    tentativa de montar a árvore completa do histórico
            \end{itemize}
            \pause
            \item \texttt{--graph}
            \begin{itemize}
                \item Desenha linhas ligando os \emph{commits} a seus pais,
                    evidenciando as ramificações e as junções
            \end{itemize}
        \end{itemize}
    \end{itemize}
\end{slide}

\section{Visualizando Diferenças}
\begin{slide}
    \begin{itemize}
        \item Para arquivos textuais, o comando \texttt{git diff} permite
            mostrar a diferenças entre versões
        \pause
        \item \texttt{git diff}
        \begin{itemize}
            \item Sem argumentos, mostra a diferença entre o estado atual da
                pasta e o índice
        \end{itemize}
        \pause
        \item \texttt{git diff --cached}
        \begin{itemize}
            \item Mostra a diferença entre o índice e o \emph{commit} atual
        \end{itemize}
        \pause
        \item \texttt{git diff commit1..commit2}
        \begin{itemize}
            \item Mostra a diferença entre dois \emph{commits}
        \end{itemize}
        \pause
        \item \texttt{git diff commit1..commit2 -- arquivo1 pasta/arquivo2}
        \begin{itemize}
            \item É possível restringir quais arquivos serão comparados
                colocando-os depois do comando e separado por dois hífens
        \end{itemize}
    \end{itemize}
\end{slide}

\section{Ponteiros Fixos}
\begin{slide}
    \begin{itemize}
        \item Ponteiros fixos podem ser criados
        \pause
        \item São chamados de rótulos (\emph{tags})
        \pause
        \item \texttt{git tag "rotulo"}
        \pause
        \item Útil para demarcar eventos importantes
        \begin{itemize}
            \pause
            \item Versões compartilhadas
            \pause
            \item Versões usadas para testes
        \end{itemize}
    \end{itemize}
\end{slide}

\section{Repositórios Remotos}
\begin{slide}
    \begin{itemize}
        \item Repositórios remotos podem ser referenciados
        \pause
        \item \emph{Commits} e ponteiros podem ser trocados entre os
            repositórios
        \pause
        \item Cada ponteiro presente no repositório remoto é copiado para o
            repositório local, prefixado com o nome do repositório
            \begin{itemize}
                \pause
                \item Exemplo: \textbf{repositorio/master}
            \end{itemize}
        \pause
        \item Comandos importantes
        \begin{itemize}
            \pause
            \item \texttt{git clone "url"}
            \begin{itemize}
                \item Cria um repositório local, copiando o repositório remoto
                    do endereço especificado
            \end{itemize}
            \pause
            \item \texttt{git remote add "nome"}\texttt{ "url"}
            \begin{itemize}
                \item Acrescenta o registro de um repositório remoto
            \end{itemize}
            \pause
            \item \texttt{git remote rm "nome"}
            \begin{itemize}
                \item Remove o registro de um repositório remoto
            \end{itemize}
            \pause
            \item \texttt{git push "repositorio"}\texttt{ "ponteiro"}
            \begin{itemize}
                \item Envia o ponteiro e os \emph{commits} necessários para o
                    repositório remoto
            \end{itemize}
            \pause
            \item \texttt{git fetch "repositorio"}
            \begin{itemize}
                \item Atualiza os ponteiros locais com o estado dos ponteiros
                    remotos e transfere para o repositório local os
                    \emph{commits} necessários
            \end{itemize}
        \end{itemize}
    \end{itemize}
\end{slide}

\section{Considerações Finais}
\begin{slide}
    \begin{itemize}
        \item \emph{Commits} curtos e específicos são melhores
        \pause
        \item Uso de ramificações para organização depende do gosto dos usuários
        \begin{itemize}
            \pause
            \item Pode-se fazer uma ramificação por conjunto de mudanças
            \pause
            \item Pode-se manter somente uma ramificação central, usando outras
                de forma temporária na máquina local
        \end{itemize}
        \pause
        \item A ferramenta não resolve o problema sozinha
    \end{itemize}
\end{slide}

\section{Próximos Passos}
\begin{slide}
    \begin{itemize}
        \item Criar um repositório na nuvem
        \begin{itemize}
            \pause
            \item GitHub \texttt{https://www.github.com/}
            \pause
            \item BitBucket \texttt{https://www.bitbucket.org}
        \end{itemize}
        \pause
        \item Ler tutoriais
        \begin{itemize}
            \pause
            \item \texttt{https://atlassian.com/git/tutorials}
            \pause
            \item \texttt{https://git-scm.com/docs/gittutorial}
        \end{itemize}
        \pause
        \item Ler livro
        \begin{itemize}
            \pause
            \item Version Control by Example
            \begin{itemize}
                \pause
                \item \texttt{http://ericsink.com/vcbe/}
            \end{itemize}
            \pause
            \item Pro Git
            \begin{itemize}
                \pause
                \item \texttt{https://git-scm.com/book/en/v2}
            \end{itemize}
        \end{itemize}
    \end{itemize}
\end{slide}

\end{document}
