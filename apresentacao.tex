\documentclass{beamer}

\usepackage[utf8]{inputenc}
\usepackage[brazil]{babel}

\mode<presentiation>{\usetheme{boxes}}

\setbeamertemplate{footline}[frame number]

\newenvironment{slide}{\begin{frame}{\insertsection}}{\end{frame}}

\title{Introdução ao Git}
\author{Janito Vaqueiro Ferreira Filho}
\institute{Universidade Estadual de Campinas}

\begin{document}

\begin{frame}
    \titlepage
\end{frame}

\begin{frame}
    \frametitle{Agenda}
    \tableofcontents
\end{frame}

\section{Introdução}
\begin{slide}
    \begin{itemize}
        \item Todo projeto contém artefatos
        \pause
        \item A complexidade de organizar esses artefatos depende de:
        \pause
            \begin{itemize}
                \item Número de artefatos
                \pause
                \item Sujeitabilidade a mudanças
                \pause
                \item Número de pessoas envolvidas
            \end{itemize}
    \end{itemize}
\end{slide}

\section{Motivação}
\begin{slide}
    A organização incorreta pode resultar em diversos problemas
    \begin{itemize}
        \pause
        \item Perda de artefatos
        \pause
        \item Perda da rastreabilidade
        \pause
        \item Perda da reprodutibilidade
        \pause
        \item Perda de coerência no uso dos artefatos entre os envolvidos
    \end{itemize}
\end{slide}

\section{Solução}
\begin{slide}
    Todas as soluções estão relacionadas a dois pontos chaves
    \begin{itemize}
        \pause
        \item Aumento da documentação
        \begin{itemize}
            \pause
            \item Rotular os artefatos
            \pause
            \item Descrever os artefatos
            \pause
            \item Registrar a evolução dos artefatos
        \end{itemize}
        \pause
        \item Aumento da redundância
        \begin{itemize}
            \pause
            \item Manter cópias dos artefatos
            \pause
            \item Manter cópias do estado dos artefatos em instantes de tempo
                \pause (versões dos artefatos)
            \pause
            \item Manter cópias da documentação e dos registros
        \end{itemize}
    \end{itemize}
\end{slide}

\section{Sistema de Controle de Versionamento}
\begin{slide}
    \begin{itemize}
        \item Ferramenta para auxiliar na organização de artefatos digitais
            \pause (arquivos)
        \pause
        \item Foco em controle de versões dos artefatos
        \pause
        \item Foco em facilitar o uso por diversas pessoas
        \pause
        \item \textbf{Repositórios} são conjuntos de artefatos e registros de um
            mesmo projeto organizados de forma unificada
    \end{itemize}
\end{slide}

\section{Características do Git}
\begin{slide}
    \begin{itemize}
        \item Descentralizado
        \begin{itemize}
            \pause
            \item Cada pessoa (ou computador) contém uma cópia completa do
                repositório
            \pause
            \item Não requer conexão à internet
        \end{itemize}
        \pause
        \item Versionamento do conjunto de artefatos como um todo
        \begin{itemize}
            \pause
            \item Não há necessidade de controlar o versionamento de cada
                arquivo de forma individual
        \end{itemize}
        \pause
        \item Projetado para melhor funcionamento com arquivos textuais
        \begin{itemize}
            \pause
            \item Algumas funcionalidades são perdidas se arquivos binários
                forem utilizados \pause (.exe\pause, .pdf\pause, .zip\pause,
                ...)
        \end{itemize}
    \end{itemize}
\end{slide}

\section{Conceitos Básicos do Git}
\begin{slide}
    \begin{itemize}
        \item O Git controla um diretório contendo o repositório
        \begin{itemize}
            \pause
            \item O comando \texttt{git init} cria um repositório em uma pasta
        \end{itemize}
        \pause
        \item Dentro da pasta, ele cria uma pasta oculta \emph{.git}
        \begin{itemize}
            \pause
            \item Cópias dos artefatos, dos registros e arquivos internos
        \end{itemize}
        \pause
        \item O Git irá controlar os arquivos da pasta que são artefatos
        \begin{itemize}
            \pause
            \item O arquivo \emph{.gitignore} pode ser usado para definir quais
                arquivos não são artefatos
            \begin{itemize}
                \pause
                \item Arquivos temporários
                \pause
                \item Arquivos gerados a partir dos artefatos
                \pause
                \item Arquivos locais (específicos ao computador atual)
                \pause
                \item Arquivos de segurança (com senhas, por exemplo)
            \end{itemize}
        \end{itemize}
    \end{itemize}
\end{slide}

\section{Controlando o Diretório}
\begin{slide}
    \begin{itemize}
        \item Como o Git controla o diretório do repositório, é preciso ter
            alguns cuidados
        \begin{itemize}
            \pause
            \item Pedir para Git voltar para uma versão anterior sobrescreve os
                artefatos
            \pause
            \item A ideia é trazer a pasta para o estado anterior exatamente
                como ela estava
            \pause
            \item Se as mudanças não forem versionadas antes de realizar este
                tipo de operação, alterações podem ser perdidas
            \pause
            \item Se em uma versão anterior um dado artefato não existia, o
                arquivo será apagado quando se voltar àquela versão
        \end{itemize}
    \end{itemize}
\end{slide}

\section{Índice}
\begin{slide}
    \begin{itemize}
        \item Um local temporário de cópias de artefatos
        \pause
        \item Os arquivos no índice serão usados para formar uma nova versão
        \pause
        \item Comandos importantes
        \begin{itemize}
            \pause
            \item \texttt{git status}
            \pause
            \item \texttt{git add "arquivo"}
            \pause
            \item \texttt{git reset -- "arquivo"}
            \pause
            \item \texttt{git rm "arquivo"}
            \pause
            \item \texttt{git diff --cached}
        \end{itemize}
    \end{itemize}
\end{slide}

\section{\emph{Commit}}
\begin{slide}
    \begin{itemize}
        \item Representa uma versão do repositório
        \pause
        \item Referencia uma cópia do estado de todos os artefatos em um
            determinando instante de tempo
        \pause
        \item Informações que compõem um \emph{commit}
        \begin{itemize}
            \pause
            \item Cópia dos artefatos \pause (estado)
            \pause
            \item Título e descrição \pause (porquê)
            \pause
            \item Autor \pause (quem)
            \pause
            \item Data \pause (quando)
            \pause
            \item \emph{Commit} pai \pause (rastreabilidade)
        \end{itemize}
        \pause
        \item Comandos importantes
        \begin{itemize}
            \pause
            \item \texttt{git commit}
            \pause
            \item \texttt{git show}
        \end{itemize}
    \end{itemize}
\end{slide}

\end{document}
